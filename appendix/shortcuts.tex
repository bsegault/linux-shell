\section{Notation des raccourcis} \label{appendix:shortcuts}
Dans ce document, la notation des raccourcis et des touches est "naturelle", c'est à dire que chaque touche est mentionnée par son nom (tel qu'affiché sur la plupart des claviers), la combinaison de touche simultanée est marquée par un signe plus (\textbf{\texttt{+}}) tandis que la combinaison successive est marquée par une virgule (\textbf{\texttt{,}}). Cette notation n'est pas toujours utilisée, notamment à l'international. Le tableau \ref{tab:shortcuts_symbols} présente les deux notations.

\begin{table}[h!]
    \centering
    \begin{tabularx}{\textwidth}{| c | c | X |}
        \hline
        \textbf{Naturelle}& \textbf{Contractée} & \textbf{Symbole, signification et touche associée} \\ \hline
        \multicolumn{3}{|c|}{\textbf{Touches modificatrices}}  \\ \hline
        \texttt{Ctrl}   & \texttt{C} ou \texttt{\^} &
            Touche modificatrice "Contrôle" (\textit{Control modifier key}) \\ \hline
        \texttt{Alt}    & \texttt{M} &
            \utfcode{2387} ou \utfcode{2325} Touche modificatrice "Alternative" \newline
            Touche Option sur un clavier Apple (\textit{Alt modifier key}) \\ \hline
        \texttt{Meta}   & \texttt{} &
            Touche Windows \newline
            \utfcode{2318}Commande sur un clavier Apple (\textit{Meta key}, \textit{Super key}) \\ \hline
        \texttt{Maj}    & \texttt{} &
            \utfcode{21E7} Changement de casse temporaire (\textit{Shift key}) \\ \hline \hline

        \multicolumn{3}{|c|}{\textbf{Touches de contrôles}} \\ \hline
        \texttt{Echap}  & \texttt{M} &
            Échappement (\textit{Escape key}) \\ \hline
        \texttt{Entrée} & \texttt{RET} &
            \utfcode{21B5} Retour à la ligne(\textit{Enter key} ou \textit{Return key}) \\ \hline
        \texttt{Retour} & \texttt{BS} ou \texttt{BKSP} &
            \utfcode{232B} Retour arrière (suppression à gauche) (\textit{Backspace key}) \\ \hline
        \texttt{Suppr}  & \texttt{DEL} &
            Suppression à droite (\textit{Delete key}) \\ \hline \hline

        \multicolumn{3}{|c|}{\textbf{Symboles de combinaison de touches}} \\ \hline
        \texttt{+}      & \texttt{-} ou rien &
            Combinaison simultanée \\ \hline
        \texttt{,}      & \texttt{\textvisiblespace} (espace) &
            Combinaison successive \\ \hline
      \end{tabularx}
    \caption{Notations naturelle française et contractée} \label{tab:shortcuts_symbols}
\end{table}

\note{Note:} Très souvent, \texttt{M-}\textit{Touche} peut être assimilé à \texttt{Echap,} \textit{Touche}, ou bien à \texttt{Alt +} \textit{Touche}. Ainsi, \texttt{M-X} peut signifier aussi bien \say{Appui simultané sur \texttt{Alt} et X} que \say{Appui sur \texttt{Echap}, puis appui sur \texttt{X}}.

À titre d'exemple, les notations suivantes sont équivalentes entre elles :
\begin{itemize}
    \item \texttt{Ctrl+C} et \texttt{\^{}C} signifient \say{Appui simultané sur les touches \texttt{Ctrl} et C},
    \item \texttt{Alt+T} et \texttt{M-T} signifient \say{Appui simultané sur les touches \texttt{Alt} et T},
    \item \texttt{Ctrl+P, Alt+X} et \texttt{C-P M-X} signifient \say{Appui simultané sur les touches \texttt{Ctrl} et P, puis appui simultané sur \texttt{Alt} et X}.
\end{itemize}
