\section{\texttt{emacs} Cheat Sheet} \label{appendix:emacs} \command{emacs}

Emacs est un éditeur de texte en ligne de commande aux fonctionnalités très riches. Il dispose de tous les outils que l'on peut s'attendre à trouver sur un éditeur de texte avancé, et plus encore. Il fait partie du projet GNU et est donc maintenu par la fondation du même nom.

Son fonctionnement est souvent considéré comme plus intuitif que \cmdref{vim} car, dès son lancement, il permet de saisir du texte au clavier, sans avoir à passer dans un mode d'édition. Il dispose de nombreux raccourcis claviers, mais ceux-ci diffèrent totalement des raccourcis usuels.

Un autre aspect particulier d'\cmdref{emacs} est qu'il dispose d'un système de \textit{plugins}. De nombreux paquets sont référencés sur \href{https://github.com/emacs-tw/awesome-emacs}{un dépôt tiers}. Ceux-ci permettent de rajouter des fonctionnalités diverses et variés, qui peuvent étendre le logiciel à bien plus qu'un éditeur de texte. Des plugins pour la gestion de tâches, de \textit{playlists} musicales voire de navigateur Internet existent !

\note{Notation des raccourcis :} La notation des raccourcis de ce document est naturelle et française. Elle n'est donc que rarement utilisée dans des documentations, supports ou aide en ligne. L'annexe \ref{appendix:shortcuts} évoque les différentes notations.

Les raccourcis essentiels pour prendre cet éditeur en main sont présentés dans le tableau \ref{tab:emacs}.

\begin{table}[h!]
    \centering
    \begin{tabularx}{\textwidth}{| c | X |}
        \hline
            \multicolumn{2}{|c|}{\textbf{Contrôle de l'éditeur}} \\ \hline
            \textbf{Raccourci}& \textbf{Signification} \\ \hline
        \texttt{Ctrl + X, Ctrl + F}  & Ouvrir un fichier. \\ \hline
        \texttt{Ctrl + X, Ctrl + S}  & Sauvegarder le fichier courant. \\ \hline
        \texttt{Ctrl + X, S}  & Sauvegarder tous les fichiers ouverts. \\ \hline
        \texttt{Ctrl + Z}  & Mettre Emacs en pause pour reprendre plus tard (voir \cmdref{fg}). \\ \hline
        \texttt{Ctrl + X, Ctrl + C}  & Quitter Emacs. \\ \hline \hline
        
        \texttt{Ctrl + X, 2}  & Divise la fenêtre courante en deux, verticalement. \\ \hline
        \texttt{Ctrl + X, 3}  & Divise la fenêtre courante en deux, horizontalement. \\ \hline
        \texttt{Ctrl + X, 1}  & Ferme les autres mini-fenêtres Emacs. \\ \hline \hline
        
        \texttt{Ctrl + X, U} ou \texttt{Ctrl+ \_}  & Annule les dernier changement. \\ \hline
        \texttt{Echap, X, revert-buffer}  & Annule tous les changements effectués. \\ \hline
        
        \nocell{2}
        \multicolumn{2}{|c|}{\textbf{Déplacements}} \\ \hline
        \textbf{Raccourci}& \textbf{Signification} \\ \hline
        \texttt{Echap, G, G}  &  Aller à une ligne spécifique. \\ \hline
        \texttt{Alt + B}  &  Aller au mot suivant. \\ \hline
        \texttt{Alt + F}  &  Aller au mot précédent. \\ \hline
        \texttt{Alt + M}  &  Aller au début de la ligne (prend en compte l'indentation).\\ \hline
        \texttt{Echap, G, G}  &  Aller à une ligne spécifique\\ \hline
        
    \end{tabularx}
    \caption{Raccourcis clavier essentiels de \cmdref{emacs}} \label{tab:emacs}
\end{table}

\newpage
Sous Emacs, un fichier ouvert est appelé un \textit{Buffer}. Les commandes tapées sont visibles dans une zone dédiée (similaire à celle de \cmdref{vim}) qui s'appelle le \textit{minibuffer}. Commencer une combinaison de touches active le \textit{minibuffer}, qui dispose de règles particulières.

Le tableau \ref{tab:emacs_advanced} référence des commandes utiles qui font usage du \textit{minibuffer}.

\begin{table}[h!]
    \centering
    \begin{tabularx}{\textwidth}{| c | X |}
        \hline
        \multicolumn{2}{|c|}{\textbf{Aide}} \\ \hline
        \textbf{Raccourci}& \textbf{Signification} \\ \hline
        \texttt{Ctrl + H, ?} & Affiche l'aide générale d'Emacs \\ \hline
        \texttt{Ctrl + H a} \textit{T} & Permet d'afficher l'aide de la commande associée à la touche \textit{T}. \\ \hline
        
        \multicolumn{2}{|c|}{\textbf{Recherche}} \\ \hline
        \textbf{Raccourci}& \textbf{Signification} \\ \hline
        \texttt{Ctrl + S} & Recherche de texte. Répéter déplace à l'occurrence suivante. \\ \hline
        \texttt{Ctrl + R} & Recherche de texte en arrière. Répéter déplace à l'occurrence précédente. \\ \hline
        \texttt{Ctrl + Alt + S} & Recherche une regex. Répéter déplace à l'occurrence suivante. \\ \hline
        \texttt{Ctrl + Alt + R} & Recherche une regex en arrière. Répéter déplace à l'occurrence précédente. \\ \hline
        \texttt{Alt + P}  & Reprend le terme de recherche précédent. \\ \hline
        \texttt{Alt + N}  & Reprend le terme de recherche suivant. \\ \hline
        \texttt{Entrée}  & Quitte le mode de recherche (et le \textit{minibuffer}). \\ \hline
        
        \nocell{2}
        \multicolumn{2}{|c|}{\textbf{Remplacement}} \\ \hline
        \textbf{Raccourci}& \textbf{Signification} \\ \hline
        \texttt{Alt + \%} & Recherche d'une chaîne pour la remplacer. \newline
            Taper la chaîne à remplacer, \texttt{Enter}, puis la chaîne de remplacement. \\ \hline
        \texttt{Espace} ou \texttt{Y} & Remplace la chaîne sélectionnée, passe à la suivante. \\ \hline
        \texttt{Suppr} ou \texttt{N} & Ignore la chaîne sélectionnée, passe à la suivante. \\ \hline
        \texttt{Entrée}  & Quitte le mode de recherche (et le \textit{minibuffer}). \\ \hline
        
        \nocell{2}
        \multicolumn{2}{|c|}{\textbf{Copie}} \\ \hline
        \textbf{Raccourci}& \textbf{Signification} \\ \hline
        \texttt{Maj+ \texttt{\textleftarrow} \texttt{\textdownarrow} \texttt{\textuparrow} \texttt{\textrightarrow}} & Séléction de texte\\ \hline
        \texttt{Alt + W} & Copie le texte sélectionné. \\ \hline
        \texttt{Ctrl + W}& Coupe le texte sélectionné. \\ \hline
        \texttt{Ctrl + Y}& Colle le texte copié. \\ \hline
            
    \end{tabularx}
    \caption{Raccourcis clavier avancés de \cmdref{emacs}} \label{tab:emacs_advanced}
\end{table}

Les raccourcis présentés ici ne sont qu'un condensé des possibilités d'Emacs. Une présentation plus complète peut être trouvée sur la \href{https://www.gnu.org/software/emacs}{page officielle du logiciel}, tandis qu'un résumé moins succinct des raccourcis clavier se trouve sur \href{https://www.gnu.org/software/emacs/refcards/pdf/refcard.pdf}{référence d'utilisation}.
