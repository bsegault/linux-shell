\section*{Introduction}
\addcontentsline{toc}{section}{Introduction}

\subsection*{Objectif de ce document}
L'objectif de ce document est de servir d'introduction détaillée au monde de Linux et de Bash afin de comprendre plus en détail le fonctionnement des systèmes Linux et de Bash.

Ce document va plus loin que les objectifs du module, notamment pour servir de transition entre les modules de PFSI, de C et de RS de TELECOM Nancy. Il n'a donc pas pour but d'être retenu intégralement par coeur.

\note{Retours et commentaires :} Tous types de retours concernant ce document sont appréciés. N'hésitez pas à me faire part de toute erreur ou manque de clarté que vous auriez pu remarquer, et à proposer vos suggestions :
\begin{itemize}
    \item en créant une \textit{issue} sur le \href{https://github.com/bsegault/shell}{dépôt GitHub},
    \item par mail à l'adresse \email{benjamin.segault@telecomnancy.net}.
\end{itemize}

\subsection*{Pourquoi apprendre Bash ?}

La maîtrise d'un langage de programmation est toujours un atout. Bash en est un, avec une place toute particulière. En effet, il s'agit d'un composant central des systèmes Linux et basés sur UNIX et s'utilise de plus directement au sein du terminal, outil central pour tout ingénieur informatique, que ce soit sur un environnement de bureau ou en interaction avec un serveur. \newline
Avec son arrivée sous Windows, la majorité des systèmes d'exploitation peuvent bénéficier de Bash : Linux, macOS, Windows et même Android. De plus, avec l'avènement de l'Internet des objets et des mini-ordinateurs tels que le Raspberry Pi, l'usage du terminal est essentiel pour piloter ces appareils qui n'ont pas toujours d'écrans.

L'interface du terminal propose une interaction rapide avec le système, mais nécessite une approche particulière et un apprentissage légèrement différent des autres langages de programmation. Car, pour bien utiliser Bash, il faut bien comprendre le fonctionnement d'un système de type UNIX. Ce mode de réflexion et cette approche différente de la programmation permet de s'ouvrir plus facilement à d'autres systèmes en utilisant un langage puissant et généraliste.

\subsection*{Objectif de ce document}
\begin{itemize}
    \item Comprendre l'environnement UNIX et l'utiliser,
    \item Avoir les bases en Bash pour réaliser des scripts et manipuler le système,
    \item Savoir quand utiliser Bash.
\end{itemize}

Sans être un objectif à part entière, un des enjeux de ce module est également de faciliter l'usage de l'environnement Linux pour le module de C.